\section{Coté serveur}
Le coté serveur est celui dont les outils necessaires sont integré dans l'application lui même. Nous verons d'avantages à ce sujet dans la partie qui concerne la structure et les modules du serveur. Ce sont pour la majorité des traducteurs, des controlleurs et des surveillants. ls permettent de connaitre l'etat de l'application, les details de son fonctionnement et l'etat des fichiers qui composent l'application.
Mis à part ces modules les applications suivantes sont installé indépendament.
\subsection{Atom}
Editeur de text libre
\subsection{Curl}
Générateur de requetes pour serveur. C'est un programme très utile pour simuler des requêtes pour le serveur. En effet certines requetes comme la methode \textit{POST} envoie des données contenu dans la requete. La meilleure façon de simuler ce genre d'actionest d'utiliser \textit{CURL}

C'est un programme libre, disponible sur plusieurs plateforme: WIN, OSX, UNIX, ...

Sur une plateforme windows les procédures d'installation, de paramètrage et d'utilisation sont lzs suivantes: