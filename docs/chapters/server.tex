\chapter{Partie serveur}
Le serveur est basé sur NodeJs. Comme framework nous emploieront \textbf{Express}. C'est le framework le plus maléable pour une architecture en MVC. Nous verons plus en details ses fonctionnalité, surtout, celles qui sont utilisé dans la realisation de ce projet.
Notre serveur est en mesure de repondre à une requête de deux manières différentes:
\begin{itemize}
	\item renvoie des \textit{pages pré-fabriqué} depuis le serveur
	\item retourne des \textit{données brutes ou semi- traités}
\end{itemize}
Cette partie se concentrera donc sur:
\begin{itemize}
	\item La structure d'un serveur d'application et serveur web en générale puis nous verons une squelette de serveur sous NodeJs. Cela permettra d'avoir une certaine comprehension du mode de donctionnement du serveur dans ses bases bien que cela ne montrera pas toutes les fonctionnalités possibles.
	\item L'importance de l'utilisation d'un \textit{framework} dans la bien tenu du projet. Cela suivi par la présentations de quelques frameworks, les plus utilisés et les plus fiables.
	\item Les fonctions basiques qu'offre le framework \textit{Express}. Nous nous axerons sur les fonctions que nous allons utiliser dans le projet.
	\item Les fonctions du serveur. Dans cette partie nous expliquerons comment le serveur gère les requêtes qu'il reçoit. Nous verons aussi les fonctions concernant la génération de pages, la persistance, ...
\end{itemize}

